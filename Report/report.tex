\documentclass[11pt,a4paper]{report}
\usepackage[utf8x]{inputenc}
\usepackage{graphicx}
\usepackage[labelfont=bf]{caption}
\usepackage{float}
\usepackage{hyperref}

\graphicspath{{C:/Users/achantreau/Documents/GitHub/BEng-Individual-Project/Report/images/}}

\usepackage{fancyhdr}
\setlength{\headheight}{30pt}
\pagestyle{fancy}

\renewcommand{\chaptermark}[1]{\markboth{#1}{}}
\renewcommand{\sectionmark}[1]{\markright{#1}{}}

\fancyhf{}
\lhead{\fancyplain{}{\thepage}}
\chead{}
\rhead{\fancyplain{}{\textit{\leftmark}}}
\rfoot{\thepage}

\setlength\parindent{0pt}
\setcounter{page}{1}
\pagenumbering{roman}

\newcommand{\HRule}{\rule{\linewidth}{0.5mm}}

\begin{document}

%-----------------------------------------------------------
% TITLE SECTION
%-----------------------------------------------------------
\begin{titlepage}
\begin{center}

\textsc{\LARGE Imperial College London}\\[1.5cm]

\textsc{\Large BEng Individual Project - Final Report}\\[0.5cm]

% Title
\HRule \\[0.4cm]
{ \huge \bfseries jSCAPE - Java Self-assessment Center of Adaptive Programming Exercises \\[0.4cm] }

\HRule \\[1.5cm]

% Author and supervisor
\begin{minipage}{0.4\textwidth}
\begin{flushleft} \large
\emph{Author:}\\
Alexis \textsc{Chantreau}
\end{flushleft}
\end{minipage}
\begin{minipage}{0.4\textwidth}
\begin{flushright} \large
\emph{Supervisor:} \\
Dr.~Tristan \textsc{Allwood}
\end{flushright}
\end{minipage}

\vfill

% Bottom of the page
{\large \today}

\end{center}
\end{titlepage}

%-----------------------------------------------------------
% ABSTRACT SECTION
%-----------------------------------------------------------
%\begin{abstract}\centering
%
%\end{abstract}

%-----------------------------------------------------------
% TABLE OF CONTENTS SECTION
%-----------------------------------------------------------
\tableofcontents

\newpage
\setcounter{page}{1}
\pagenumbering{arabic}

%-----------------------------------------------------------
% LIST OF FIGURES SECTION
%-----------------------------------------------------------
\listoffigures

%-----------------------------------------------------------
% INTRODUCTION SECTION
%-----------------------------------------------------------
\chapter{Introduction}

\section{Motivation}
Programming is generally acknowledged to be a difficult discipline to learn, requiring problem solving skills, attention to detail and the ability to think abstractly. One could say that these skills are somewhat developed in high school during mathematics courses, but programming is still a ``beast" of its own. In addition, different programming paradigms exist, such as functional programming or imperative programming. Knowing how to program in Java can still make learning Haskell a difficult process.\newline

Yet, in this 21st century society, programming is definitely a useful skill to have, and there is an interest in the population to learn these skills. Indeed, this can be seen by the number of students enrolling in computer science courses at universities, or the increase in websites such as Codecademy\cite{Codecademy}, Coursera\cite{Coursera}, Udacity\cite{Udacity}, which provide online computer science/programming courses for free.\newline

In many of these situations, it is difficult for teachers, lecturers or course leaders to provide enough support through exercises, assignments and to monitor student's progress in such a way which allows them to modify their teaching to help struggling students. Indeed, teachers and students can generally only rely on a few homework assignments to get an idea of how they are doing. Coming up with a solution to increase the amount of practise and feedback would be beneficial to both students and teachers. \newline

Thus, there is a clear need to provide a platform for students to practise their programming skills and understanding of programming concepts, in a context of self-assessment only. In such a system, requiring teachers to come up with all the exercises by themselves can be both time consuming and ineffective: some exercises may not be challenging enough for certain top students, or on the contrary too difficult for struggling students, which can be discouraging for them.

\section{Objectives}
Having identified the problems associated with teaching and learning programming, we were led to formulating objectives in order to make the project successful and useful to the parties involved. \newline

The main objective of the project was to produce a web-based application to be used in self-assessing one's programming knowledge, whether it be in high school, at university or as part of an online course. For this application, four key features were identified:

\begin{itemize}
\item \textbf{Programming questions/exercises -} The web platform should allow students to practise their programming skills and understanding of programming concepts. There should be no limit to the number of questions a student can answer, so that if a student desires more practice, then he should be able to do that. Additionally, it should be possible for a specific set of people, such as teachers, lecturers or tutors, to add questions/exercises to the system.
\item \textbf{Progress tracking -} Designated people, such as teachers, lecturers or tutors, should have access to detailed statistics about the students performances. This will provide them with useful information about difficulties particular students, or the entire class, may be facing. In addition, the system should give feedback to the students, in the form of simple statistics, allowing them to identify their weak areas and thus improve on them.
\item \textbf{Adaptive difficulty -} The questions or exercises presented to the students should be suited to their ability. Not only will this stimulate the learning process, but it will also give a better indication of a student's understanding of the programming concepts being tested.
\item \textbf{Automated generation -} There should be a mechanism to allow for some degree of automated or semi-automated generation of exercises. This will provide a large supply of ``fresh" questions, so that students don't end up answering the same questions and memorizing the answers to them.
\end{itemize}

While investigating existing solutions (chapter \ref{chap:related-work}) we found out that some of these features were less common than others. The availability of programming exercises and progress tracking are very essential in such systems, therefore many of the related software we looked at implemented these features. On the other hand, relatively few tools integrated some form of adaptive difficulty. Finally, almost none of the tools featured automated generation of questions, opting instead to allow exercises to be added manually to the system, or downloaded from existing exercise banks.

\section{Contributions}
Within the context given above, this project makes the following contributions:
\begin{itemize}
\item \textbf{jSCAPE}: a web application for students, named Java Self-assessment Center of Adaptive Programming Exercises, with the following features:
      \begin{itemize}
      \item[-] the ability to view programming exercises and answer them while receiving feedback, a so called form of self-assessment.
      \item[-] graphs, tables and pie charts displaying statistical data on the student's performance.
      \item[-] three different exercise selection algorithms, that decide which is the best exercise to present to the student.
      \end{itemize}
\item \textbf{Admin tool}: a tool for teachers/lecturers/tutors for:
      \begin{itemize}
      \item[-] displaying student performance statistics.
      \item[-] displaying exercise statistics.
      \item[-] defining exercise categories.
      \item[-] adding exercises manually.
      \item[-] generating exercises automatically.
      \end{itemize}
\end{itemize}

\section{Report Structure}
Chapter \ref{chap:background} will present the theoretical basis for this project and the concepts necessary to follow the implementation details of the system.\newline

Chapter \ref{chap:related-work} will give an overview of related work, and will mention how these influenced the design of jSCAPE, in particular which features would be useful for such as system.\newline

Chapter \ref{chap:jscape-system} will present the jSCAPE system in detail, showing all the features which are available.\newline

Chapter \ref{chap:implementation} will cover the design and implementation details of the jSCAPE system, in particular how the difficulty of exercises is adapted to the student's ability and how exercises can be automatically generated.\newline

Chapter \ref{chap:evaluation} will contain the results of evaluating jSCAPE as a whole, and its different components. \newline

Chapter \ref{chap:conclusion} summarises the achievements of the project and discusses possible extensions that can be made to the system in the future.


%-----------------------------------------------------------
% BACKGROUND SECTION
%-----------------------------------------------------------
\chapter{Background}
% Add maximum likelihood background explanation
% Add Bayesian probability background explanation
%Add note to say that we use the word examinee to mean someone taking a test although the test may not be an exam per say.

\section{Computer Based Tests}
CBT abbreviation
- offers advantages such as being able to display higher quality visuals such as pictures, videos, graphs, etc...
- low paperwork, everything is stored on the computer
- automatic grading, less work for teachers
- generation of statistics is made easier by the fact that the data can be processed by the computer
- nowadays a lot of learning is done on computer systems, and children are used to dealing with computers so assessment through this medium is advantageous. \newline

CBTs are typically "fixed-item" tests where all the students answer the same set of questions,  usually provided by the person responsible for the assessment. This isn't ideal since students can be presented with questions that are too easy or too difficult for them to answer. Consequently, the results of the test won't give a very accurate representation of a student's ability, and for this reason, these types of tests aren't extremely useful. This problem lead to research and the development of computerized adaptive testing (CAT).

\section{Computerized Adaptive Testing}
Computerized adaptive testing (CAT), also called \textit{tailored testing}, is a form of computer-based testing which administers questions (referred to as \textit{items} in the psychometrics literature) of the appropriate difficulty by adapting to the examinee's ability.
For example, if an examinee answers an item correctly, then the next item presented will higher on the difficulty scale. On the other hand, if they answer incorrectly, they will be presented with an item lower on the difficulty scale. \newline

From an architectural perspective, a computerized adaptive test (CAT) consists of five components \cite{CAT-Framework}:

\subsubsection{1. Calibrated item pool}
An item pool is needed to store all the items available for inclusion in a test. This item pool must be calibrated with a psychometric model. During this phase, the item parameters are estimated according to the chosen model and scaled to fit with already existing items. Usually, the psychometric model employed in these systems is called Item Response Theory (IRT) (section \ref{subsec:IRT}). Calibration is a complex process, and to be done accurately it requires a considerable amount of data. Typically, it is performed by psychometricians, aided by expensive and sophisticated calibration software.

\subsubsection{2. Starting point}
Initially, when zero items have been administered, no information is known about the examinees and so the CAT is unable to estimate their ability. As a result, the item selection algorithm will fail to choose the next item to be administered.
If there is previous information available, for example an examinee's ability estimate in a closely related subject, then this can be input into the system to form the starting point configuration. Often this data isn't available or too costly to collect, so the CAT's initial ability estimate for the examinee corresponds to the mean on the ability scale - hence the first item presented will be of average difficulty.

\subsubsection{3. Item selection algorithm}
The item selection algorithm chooses the next item to present to the examinee based on the ability estimate of the examinee up to that point. Several methods exist and largely depend on the psychometric model in use. One of the most commonly used methods is the \textit{maximum information method} (section \ref{subsec:IRT}), which selects the item which maximizes the information function with respect to the estimated ability at that point.

\subsubsection{4. Scoring algorithm}
The scoring algorithm refers to the steps taken to update the examinee's ability estimate after an item has been answered. The two most commonly used methods are \textit{maximum likelihood estimation} (SECTION REFERENCE) and \textit{Bayesian estimation} (SECTION REFERENCE).

\subsubsection{5. Termination criterion}
The termination criterion specifies when the CAT should finish. For example the CAT can terminate when the change in the ability estimate after each iteration is below a certain threshold, or when time has run out, or when $N$ items have been administered, etc... Obviously, the CAT shouldn't be terminated too early, so as to allow enough time to estimate the examinee's ability with acceptable accuracy.

\begin{figure}[H]
\centering
\includegraphics[scale=1]{cat_flowchart}
\caption{Flowchart of an adaptive test. Adapted from \cite{SIETTE}.}
\label{fig:cat_flowchart}
\end{figure}

The flowchart in figure~\ref{fig:cat_flowchart} corresponds to components 2-5, and illustrates the basics of the algorithm implemented in CAT. \cite{CAT-Wiki} gives a more detailed description of the procedure:
\begin{itemize}
\item[1.] The pool of items that haven't been administered yet is searched to determine the best item to present to the examinee, according to the current estimation of his ability.
\item[2.] The chosen item is presented to the examinee, who then answers it correctly or incorrectly.
\item[3.] The ability estimate is updated, based upon this new piece of information and the previous ability estimate.
\item[4.] Steps 1–3 are repeated until a termination criterion is met.
\item[5.] The algorithm returns a final ability estimate for the examinee's performance along with a confidence level: a percentage value indicating how accurate the estimate is.
\end{itemize}

CATs offer several advantages over traditional CBTs. As a result CATs have been used in many areas\cite{CAT-Areas}, such as education, job hiring, counselling, clinical studies, etc... Since CATs administer items by adapting to the examinee's ability, the test-taking experience ends up being a more positive one. Indeed, examinees won't have to deal with answering items which are too difficult or too easy compared to their ability level, a problem which appears in traditional CBTs.\newline

In addition, by administering only those items which will yield additional information, CATs end up being more accurate in estimating an examinee's ability level. This contrasts with CBTs which usually provide the best precision for examinees of medium ability, whereas extreme scores end up being less accurate.\newline

Lastly, CATs can come up with an ability estimate much quicker and with fewer administered items when compared to traditional CBTs. Indeed, an adaptive test can typically be shortened by 50\% and still maintain a higher level of precision than a fixed version.\cite{Weiss1984}
\newline

Despite the advantages mentioned above, CATs have some limitations. A frequent complaint is that an examinee isn't allowed to go back and change his answer to a past item. This limitation exists to prevent the examinee from intentionally answering items incorrectly to make subsequent items easier, and then going back and selecting the correct answers to achieve a perfect score. For similar reasons, it isn't possible to skip items, the examinee must select an answer to move on to the next item.\newline

The second issue has to do with the items themselves. First of all, there is the need for a large bank of items to cater to all ability levels. Developing an item pool of sufficient size can be very time consuming. David J. Weiss writes in \cite{Weiss1985} that item pools with 150-200 items are to be preferred.

Secondly, for the CAT to be of good quality the item pool needs to be calibrated accurately. This requires pre-administering the items to a sizeable sample and then simultaneously estimating all the item parameters for each item. The guidelines in \cite{CAT-Primer} suggest that sample sizes may be as large as $1000$ examinees. This phase is costly, time consuming and often times simply unfeasible.\newline

Lastly, item exposure is a possible security concern. Sometimes particular items may be presented too often and become overused. This may result in examinees becoming familiar with them and sharing them to other examinees of the same ability level, thus corrupting the results of the test. This problem can be solved to some extent by modifying the item selection algorithm to include some exposure control mechanism.\newline

A brief overview of CATs was given in this section. All of these concepts will be explored in more detail in item response theory (section \ref{subsec:IRT}) and in the implementation of adaptive testing in jSCAPE (chapter XX?).

\section{Probabilistic Test Theory}
Blablablakskdjfksdjfksdsdfk stuff about probability and their usage

\subsection{Maximum Likelihood}
% Add maximum likelihood background explanation
% Conditional probability quick overview?
% Add Bayesian probability background explanation
% Bayesian networks
% Latent variables ?

IRT is based on the idea that the probability of a correct/keyed response to an item is a mathematical function of person and item parameters. 

\begin{itemize}
\item Calculates the probability of a particular student answering a specific item correctly.
\item Different IRT models: One-Parameter Logistic (1-PL), Two-Parameter Logistic (2-PL), Three-Parameter Logistic (3-PL). Refers to the number of parameters used in the model. Parameters are: 

\begin{itemize}
\item[-] item difficulty parameter (\textit{b})
\item[-] item discrimination parameter (\textit{a})
\item[-] chance/guessing parameter (\textit{c})
\end{itemize}

\item Item Characteristic Curve, i.e. probability distribution
\end{itemize}

\subsection{Bayes probability theory}

\subsection{Item Response Theory}
\label{subsec:IRT}
For these reasons, Item response theory (IRT) has seen frequent usage when it comes to CATs....We present the different models developed to predict the probability of a correct or incorrect response to a particular item.
\subsubsection{The one-parameter logistic model}
dfsdfsdf
\subsubsection{The two-parameter logistic model}
dfsdfsdf
\subsubsection{The three-parameter logistic model}
dfsdfsdf


%-----------------------------------------------------------
% RELATED WORK SECTION
%-----------------------------------------------------------
\chapter{Related Work}
%Add some context information and introduction before delving into the related work.

Web-based/Computer based education and adaptive web-based assessment systems are a ``hot" research area, and as a result, numerous tools, environments and infrastructures have emerged. There are common features to all, however some distinguish themselves by having not so common features.
Automated exercise generation in these tools is usually non-existent or very limited. Moreover, the tools are more focused on assessing students rather than self-assessment, i.e. students take tests which count towards their final grade on these systems.\newline

There are many components involved in this project, two of the more important ones are adaptive difficulty and automated generation of exercises, so there are many tools which exist which do one or the other, very rarely both.\newline

In this section we look at related software and evaluate them with respect to the objectives listed at the beginning of the development of jSCAPE.

\section{Environment for Learning to Program}
Environment for Learning to Program (ELP) is an interactive web based environment for teaching programming to first year Information Technology students at Queensland University of Technology (QUT).

\section{CourseMarker}
CourseMarker is a re-design of Ceilidh, a computer based assessment system used at the University of Nottingham for 13 years. Ceilidh was quite a complete system, providing coursework, the management of modules and the presentation of module content.

\section{Automatic Exercise Generator with Tagged Documents based on the Intelligence of Students}
The Automatic Exercise Generator with Tagged Documents based on the Intelligence of Students (AEGIS)

\section{Programming Adaptive Testing}
\begin{figure}[H]
\centering
\includegraphics[width=\textwidth,height=\textheight,keepaspectratio]{PAT_adaptive_sequence}
\caption{Adaptive sequence of questions in PAT. (Source:??)}
\label{fig:PAT_adaptive_sequence}
\end{figure}

\section{Adaptive Self-Assessment Master}
Adaptive Self-Assessment Master (ASAM) is an extension to CourseMarker, which improves upon it by administering questions which are suited to the student's ability.

\section{System of Intelligent Evaluation Using Tests for Tele-education}
The System of Intelligent Evaluation Using Tests for Tele-education (SIETTE) is a web based environment for generating and constructing adaptive tests.

\section{Summary}
We have looked at some of the relevant work in the field of computer based education and assessment. We saw that SIETTE provided many of the features we set out to replicate in jSCAPE, therefore particular parts of our implementation will be inspired by SIETTE.

%-----------------------------------------------------------
% PROJECT PLAN SECTION
%-----------------------------------------------------------


%-----------------------------------------------------------
% EVALUATION SECTION
%-----------------------------------------------------------

\chapter{Evaluation}
Mention how our developed system performs against the advantages and disadvantages of CAT.

The evaluation stage will address mostly these two aspects:
\begin{itemize}
\item The system has correctly modelled the ability of students
\item The system is useful in helping students to learn programming and helping lecturers with getting feedback on their teaching, in the form of statistics.
\end{itemize}

\section{Qualitative}
\begin{itemize}
\item Surveys to get feedback from students on interface, usability, etc...
\item 
\end{itemize}

\section{Quantitative}
\begin{itemize}
\item Statistical analysis to evaluate item calibration and modelling of student's abilities
\item 
\end{itemize}

%-----------------------------------------------------------
% CONCLUSION SECTION
%-----------------------------------------------------------
\chapter{Conclusion}
\section{Future Work}
\begin{itemize}
\item Very flexible system so other programming languages could be offered, i.e. cSCAPE, for C and hSCAPE for Haskell. 
\item Working more extensively on the adaptive component of the system, i.e. improving the algorithm which selects questions for students based on their estimated ability.
\item Extend system to allow admins to provide their own question templates, maybe come up with a template grammar which then allows questions to be automatically generated. Or at least allow pluggable function references which will be called to generate the exercise component.
\item Add support for more question types. JavaFX is very good in that sense since it can play audio clips, video clips, show animations, the webview component has endless possibilities thanks to the inclusion of Javascript.
\item Future Work as a research project vs future work as a commercial product
\item JavaFX applications are based on the model-view-controller pattern, so a nice split can be done in the code, however I only learnt about this two weeks into the project, so all the of the GUI components are created in the code as opposed to in the FXML file.
\end{itemize}

%-----------------------------------------------------------
% Bibliography
%-----------------------------------------------------------
\begin{thebibliography}{9}

\bibitem{SIETTE} Conejo, R., Guzmán, E., Millán, E., Trella, M., Pérez-de-la-Cruz, J. L., \& Rios, A. (2004). SIETTE: A Web-Based Tool for Adaptive Testing. \textit{International Journal of Artificial Intelligence in Education, 14}, 29-61.

\bibitem{CAT-Wiki} Computerized adaptive testing. \url{http://en.wikipedia.org/wiki/Computerized_adaptive_testing}. Accessed: \today.

\bibitem{CAT-Framework} Thompson, Nathan A., \& Weiss, David A. (2011). A Framework for the Development of Computerized Adaptive Tests. \textit{Practical Assessment, Research \& Evaluation}, 16(1).

\bibitem{CAT-Areas} IRT-Based CAT. \url{http://www.iacat.org/content/irt-based-cat}. Accessed: \today.

\bibitem{Weiss1984} Weiss, D. J., \& Kingsbury, G. G. (1984). Application of computerized adaptive testing to educational problems. \textit{Journal of Educational Measurement}, 21, 361-375

\bibitem{Weiss1985} Weiss, D.J. (1985). Adaptive Testing by Computer, \textit{Journal of Consulting and Clinical Psychology}. 1985, 53, 6, pp. 774-789.

\bibitem{CAT-Primer} Wainer, H., \& Mislevy, R.J. (2000). Item response theory, calibration, and estimation. In Wainer, H. (Ed.) Computerized Adaptive Testing: A Primer. Mahwah, NJ: Lawrence Erlbaum Associates.




\end{thebibliography}

%-----------------------------------------------------------
% APPENDICES
%-----------------------------------------------------------
\appendix
\chapter{User Manual}
The user manual is included in the Help Tab of the jSCAPE client application. It is not reproduced here to limit the number of pages of this report.

\chapter{Example jSCAPE exercises}
\label{chap:example-jscape-exercises}

\subsubsection{Binary tree exercise}
\lstinputlisting[language={xml}, tabsize=4, caption={Example exercise for the Binary Tree exercise category}]{\listings/binary_tree_exercise.xml}

\subsubsection{Strings exercise}
\lstinputlisting[language={xml}, tabsize=4, caption={Example exercise for the Strings exercise category.}]{\listings/strings_exercise.xml}

\subsubsection{Conditionals exercise}
\lstinputlisting[language={xml}, tabsize=4, caption={Example exercise for the Conditionals exercise category}]{\listings/conditionals_exercise.xml}




\end{document}