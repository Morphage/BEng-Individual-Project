\chapter{Introduction}

\section{Motivation}

\begin{itemize}
\item The rise in MOOCs such as Coursera, Udacity shows that there is a real interest in learning programming.
\item Programming is considered "difficult'' to learn and it can only be learnt effectively through lots and lots of practice.
\item This is even more apparent when students are introduced to other programming languages with different paradigms such as Haskell and Prolog, or more low level programming languages such as C.
\item Hence, the need to provide a platform for students to practice their programming skills and understanding of programming concepts.
\item Having a lecturer come up with all the exercises by himself can be both time consuming and ineffective: some exercises may not be challenging enough for certain top students, or on the contrary too difficult for struggling students, which can be discouraging.
\item Lecturers can only rely on a few homework assignments to get an idea of how students are doing. Having a way for lecturers to gather large amounts of data about students' performances would be beneficial. Allows for supplementary material, exercises, etc...
\end{itemize}

\section{Objectives}
Having identified the problems associated with teaching and learning programming, we were lead to formulating objectives in order to make the project successful and useful to the parties involved. \newline

The main objective of the project was to produce a web-based teaching infrastructure to complement the introductory first year programming classes. Four key features were identified:

\begin{itemize}
\item \textbf{Programming questions/exercises -} The web platform should allow students to practice their programming skills and understanding of programming concepts. There should be no limit to the number of questions a student can answer, so that if a student desires more practice, then he should be able to do that. Additionally, it should be possible for a specific set of people, such as lecturers and tutors, to add questions/exercises to the system.
\item \textbf{Progress tracking -} Designated people, such as lecturers and tutors, should have access to detailed statistics about the students performances. This will provide them with useful information about difficulties particular students, or the entire class, may be facing. In addition, the system should give feedback to the students, in the form of simple statistics, allowing them to identify their weak areas and thus improve on them.
\item \textbf{Adaptive difficulty -} The questions or exercises presented to the students should be suited to their ability. Not only will this stimulate the learning process, but it will also give a better indication of a student's understanding of the programming concepts being tested.
\item \textbf{Automated generation -} There should be a mechanism to allow for some degree of automated or semi-automated generation of exercises. This will provide a large supply of ``fresh" questions, so that students don't end up answering the same questions and memorizing the answers to them.
\end{itemize}

While investigating existing solutions (Section 2.4 - Related Work) we found out that some of these features were less common than others. The availability of programming exercises and progress tracking are very essential in such systems, therefore many of the related software we looked at implemented these features. On the other hand, relatively few tools integrated some form of adapting the questions to the students' ability. Finally, almost none of the tools featured automated generation of questions, opting instead to allow exercises to be added manually to the system.

\section{Contributions}
Within the context given above, this project makes the following contributions:

\section{Report Structure}
