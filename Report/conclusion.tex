\chapter{Conclusion}
\label{chap:conclusion}
In this chapter, we summarise the achievements of the project and discuss possible extensions that can be made to the system in the future.

\section{Summary}
The main objective of the project was to produce a web-based application to allow students to practise their programming skills and to provide teachers with the possibility of monitoring student performance. We delivered a product, in the form of jSCAPE, which we feel strongly meets the main objective of the project.\newline

To differentiate the system from other applications, where exercises are presented randomly to students, we introduced the concept of adaptive testing. We implemented two different algorithms to vary the difficulty of exercises according to the student's performance. The algorithms were successful in doing so, however they did have shortcomings. For instance, we learnt that adaptive testing using Item Response Theory is quite complex to implement in practice, and that large amounts of data are needed for the system to be effective. \newline

We realised that developing an exercise bank by only adding exercises manually was a cumbersome and time consuming process. Therefore, we introduced automated exercise generation as a feature. We showed how the jSCAPE exercise format was defined and how exercise generators can be written to automate this process. We learnt that automated generation is quite a complex task, and as such, much improvement is still possible in this area of jSCAPE.\newline

In conclusion, we believe that jSCAPE is a complete system in the sense that it implements the core features which can be useful in the setting of computer based self-assessment of programming knowledge. However, the adaptive difficulty and automated exercise generation components only give a small flavour of what is possible, and as a result, they require more work to make jSCAPE a solid implementation of these ideas.

\section{Future Work}
\label{sec:future-work}
First of all, before anything else is done, security in jSCAPE needs to be improved. Since the purpose of this project was to provide an application with adaptive and automatically generated exercises, we didn't take normal security measures during the development of the system. For instance, student passwords aren't stored securely in the database, and all communication between the client, server and the database is done in plain text. Therefore, students could intercept exercises as they are sent from server to client, which would reveal the solution of the exercise. To fix this, we would use hashing and salt for password storage, and SSL for communication. These measures should allow jSCAPE to be used in universities or schools without any cheating or hacking passwords. \newline

In addition, to help with implementing extensions a code refactoring of jSCAPE could be considered. Indeed, \textsf{JavaFX} applications are based on the model-view-controller pattern, so a nice separation of these components can be done in the code. However we only learnt about this possibility three weeks into the project, so all the of the GUI components are created in the Java code as opposed to in a separate \textsf{FXML} file, a \textsf{JavaFX} format created specifically for this purpose.\newline

The main features of jSCAPE are automated exercise generation, adaptive difficulty, exercise provision and statistical tracking. Any of these areas could be chosen to be explored in more depth. Given more time we list some of the things which could be accomplished in the future:

\begin{itemize}
\item The system we developed is language independent, thus we could add support for other programming languages. This could be done in jSCAPE itself, or by cloning it into a separate application, e.g.  cSCAPE, for C and hSCAPE for Haskell.
\item We could add support for more exercise types, such as multiple-response questions, fill-in-the-blank, or even more interactive types of exercises thanks to the possibilities offered by \textsf{JavaFX}.
\item Currently, feedback after a student has answered an exercise is very basic. The system displays whether the student got the exercise right or wrong, and it displays the solution. We could add more useful feedback to exercises. Examples of more advanced feedback on current exercises could include execution traces or animations of traversal algorithms on the binary tree, etc...
\item We could work on a more general-purpose code generator for exercises which use code snippets as exercise data. Currently code generators are written specifically for exercise categories, but it would be more useful and less time consuming to have a code generator module, where certain attributes could be specified such as number of variables, number of loops, if any, etc... This seems to be quite a complex task, and a solid implementation of this could probably be a project on its own.
\item We could approach automated generation differently, and instead develop a feature where teachers create exercise templates, perhaps by defining a template grammar and then exercises are automatically generated from these templates. This would remove the need for writing an exercise specific generator each time.
\item We could work more extensively on the adaptive component of the system, i.e. improving the algorithm which selects exercises for students based on their estimated ability. We mentioned that obtaining a large calibrated item pool wasn't feasible in the scope of this project, so this would have to be a priority if we had more time. In addition, we could use more complex IRT models which take into account the time spent on each exercise, for instance.
\item We could do some more research on what statistical data would be useful for students and teachers and implement components to display this data in jSCAPE, e.g. longest streak of correct/incorrect answers, etc...
\end{itemize}